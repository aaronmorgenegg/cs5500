\documentclass[12pt]{article}
\usepackage{amsmath}
\usepackage{graphicx}
\usepackage{hyperref}
\usepackage[latin1]{inputenc}

\title{MPI Setup}
\author{Aaron Morgenegg}
\date{09/7/18}

\begin{document}
\maketitle

OpenMPI setup instructions for linux mint 19

\begin{enumerate}
  \item Download latest stable release of openmpi \\ \\
  \indent https://www.open-mpi.org/software/ompi/v3.1/
  \item Open up the tar to desired location \\ \\
  \indent mv ~/Downloads/openmpi-3.1.2.tar.gz ~/Projects/OpenMPI \\
  \indent tar -xzf openmpi-3.1.2.tar.gz \\
  \indent rm openmpi-3.1.2.tar.gz \\
  \item Run MPI Configuration \\ \\
  \indent cd ~/Projects/OpenMPI \\
  \indent ./configure \\
  \item Build OpenMPI - this will take a while \\ \\
  \indent sudo make all install \\
  \item Run this config (https://askubuntu.com/questions/738667/problem-with-mpicc) \\ \\
  \indent sudo ldconfig
  \item Test the installation was successful \\ \\
  \indent mpiexec --version

\end{enumerate}

Simple message pasing program to demonstrate usage of MPI.

\begin{verbatim}
#include <iostream>
#include <mpi.h>

int main(int argc, char** argv) {
	// initialize MPI
	MPI_Init(&argc, &argv);

	// stores number of processes in world_size
	int world_size;
	MPI_Comm_size(MPI_COMM_WORLD, &world_size);

	// Get the rank of this process
	int world_rank;
	MPI_Comm_rank(MPI_COMM_WORLD, &world_rank);

	int data = world_rank;
	MPI_Send(
		&data, // data to send
		1, // count, or number of things passed
		MPI_INT, // datatype
		(world_rank+1)%world_size, // destination
		0, // tag of message
		MPI_COMM_WORLD // MPI communicator
	);

	MPI_Recv(
		&data, // data to recieve
		1, // count, or number of things passed
	    	MPI_INT, // datatype
		MPI_ANY_SOURCE, // source
		MPI_ANY_TAG, // tag of message
		MPI_COMM_WORLD, // MPI communicator
		MPI_STATUS_IGNORE // MPI status
	);

	std::cout << "I am " << world_rank << " of " << world_size <<
	", recieving a message from number " << data << std::endl;

	// Finalize the MPI environment.
	MPI_Finalize();
	return 0;
}

\end{verbatim}

\end{document}

